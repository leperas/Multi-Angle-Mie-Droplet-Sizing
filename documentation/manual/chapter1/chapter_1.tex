\chapter{Introduction}
\section{Motivation}

There are many interesting engineering flows where in order to improve performance, it is important to be able to measure the size and size distribution of droplets or particles; for instance characterization of a fuel spray or a fuel jet atomizing into the flow of air through an engine.  Even though numerous excellent methods exist within the literature and many commercial systems are available which can give this information, it is likely impossible to have one system which is perfect for use in all occasions.

The work herein details a system based on planar multi-angle Mie scattering.  Sizing information gained from this technique consists of a mean droplet diameter and droplet distribution estimates for every individual point within a planar area of interest.  Only characteristics of the entire distribution at each location are measured, however the advantage is that when compared to some well known techniques, such as Malvern or Phase Doppler Particle Anemometry (PDPA), spatially accurate results for an entire flow-field may be obtained quickly and relatively economically.

\section{Scope/Objectives}

The current work explores the use of Mie scattering from an entire 2D plane to obtain information about the size and size distribution of particles in a typical spray.  The planar method makes acquisition of data within a large field of interest possible, while the use of relatively inexpensive instrumentation makes the method available to a large number of research situations and facilities.  
\newpage
The result of this current work is:
\begin{compactitem}
\item presentation of a relatively inexpensive method which can produce spatially accurate information about the size of droplets in a spray, and
\item an exploration of the abilities and limitations of the developed method.  
\end{compactitem}
\vspace*{0.15in}

The following section briefly reviews sprays of the type concerned in this work, reviews past and current optical sizing methods, overviews the general Mie theory of scattering from small particles, and specifically analyzes Mie theory as it will relate to the planar scattering method developed within this work.

\section{Particle Sizing Background}
\label{background}

%\input{./chapter1/sprays_distributions/spray_dist.tex}

%\input{./chapter1/sizing_methods/sizing_methods.tex}

%\input{./chapter1/mie_scattering/mie_scattering.tex}

